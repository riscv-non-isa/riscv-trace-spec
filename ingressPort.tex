\chapter{Ingress Port} \label{Interface}

\section{Instruction Interface}
This section describes the interface between a RISC-V core and the
trace encoder. The trace interface conveys information about
instruction-retirement and exception events.

\begin{table}[htp]
    \centering
    \caption{Core-Encoder signals}
    \label{tab:ingress}
    \begin{tabulary}{\textwidth}{|l|p{80mm}|}
        \hline
        \textbf {Signal} & \textbf {Function} \\
        \hline
        \textbf {iretire} [\textit{iretire\_width\_p}-1:0] & Number of halfwords represented by instructions retired in this block\\
        \hline
        \textbf {ntkn} [\textit{ntkn\_width\_p}-1:0] & Number of nontaken branches in this block\\
        \hline
        \textbf {itype} [\textit{itype\_width\_p}-1:0] & Termination type of the instruction block\newline
        0: Final instruction in the block is none of the other named \textbf{itype} codes\newline
        1: Exception. An exception occurred following the final retired instruction in the block\newline
        2: Interrupt. An interrupt occurred following the final retired instruction in the block\newline
        3: Exception return\newline
        4: Reserved\newline
        5: Reserved\newline
        6: Taken branch\newline
        7: Reserved\newline
        8: Unpredictable call\newline
        9: Predictable call\newline
        10: Unpredictable return\newline
        11: Predictable return\newline
        12: Unpredictable non call/return jump\newline
        13: Predictable non call/return jump\\
        \hline
        \textbf {cause} [\textit{context\_width\_p}-1:0] & Exception or interrupt cause (\textbf{\textit{mcause/scause}}), \newline
        Ignored unless \textbf {itype}=1 or 2\\
        \hline
        \textbf {tval}[\textit{iaddress\_width\_p}-1:0]& The associated trap value, e.g., the
    faulting virtual address for address exceptions, as would be
    written to the \textbf {mtval/stval} CSR. Future optional extensions may define \textbf {tval} to provide ancillary information in cases where it currently supplies zero\newline
    Ignored unless \textbf {itype}=1 or 2\\
        \hline
        \textbf {priv}[\textit{privilege\_width\_p}-1:0] & Privilege level for all instructions in this block\\
        \hline
        \textbf {context}[\textit{context\_width\_p}-1:0] & Context and/or Hart ID for all instructions in this block\\
        \hline
        \textbf {iaddr} [\textit{iaddress\_width\_p}-1:0] & The address of the 1st instruction retired in this block.\newline
        Invalid if \textbf{iretires}=0 \\
        \hline
        \textbf {ilastsize}[\textit{ilastsize\_width\_p}-1:0] & The size of the last retired instruction. For cases where the address of the last retired instruction is needed\\
        \hline
        \textbf {context\_type}[\textit{context\_width\_p}-1:0] & Behavior type of \textbf {context}\newline
        0: Context change with dicontinuity\newline
        1: Precise context change\newline
        2: Imprecise context change\newline
        3: Notification\\
        \hline
    \end{tabulary}
\end{table}

Table~\ref{tab:ingress} lists the signals in the interface. The
information presented on the ingress port represents a contiguous
block of instructions starting at \textbf{iaddr}, all of which retired
at the same cycle. Note if \textbf{itype} is 1 or 2 (indicating an
exception or an interrupt), the number of instructions retired may be
zero. \textbf{cause} and \textbf{tval} are only defined if
\textbf{itype} is 1 or 2. is \textbf {iretire}=0 and \textbf{itype}=0,
the values of all other signals are undefined.


\textbf {iretire} contains the number of halfwords represented by
instructions retired in this bundle. Half words rather than
instruction count enables the encoder to easily compute the 'from'
address of a branch or exception without having access to opcode.

If address translation is enabled, \textbf {iaddr} is a virtual
address, else it is a physical address. Virtual addresses narrower
than \textit{iaddress\_width\_p} bits must be sign-extended to make
computation of differential addresses easier, and physical addresses
narrower than \textit{iaddress\_width\_p} bits must be zero-extended.

For cores that can retire a maximum of N taken branches per clock
cycle, the signal group (\textbf{iretire}, \textbf{itype},
\textbf{ntkn}) must be replicated N times. Signal group 0
represents information about the oldest instruction block, and group N-1
represents the newest instruction block. The interface supports no more
than one exception or interrupt per cycle and so \textbf{cause} and
\textbf{tval} are not replicated. Futhermore, \textbf{itype} can only
take the value 1 or 2 in one of the signal groups, and this must be
the newest valid group (i.e. \textbf{iretires} and \textbf{itype} must
be zero for higher numbered groups). If fewer than N taken branches
are retired in a cycle, then lower numbered groups must be used
first. For example, if there is one taken branch, use only group 0, if
there are two taken branches, instructions upto the 1st taken branch
must be reported in group 0 and instructions upto the 2nd taken branch
must be reported in group 1 and son on.

Table~\ref{tab:itype} specifies the instructions for the various \textbf{itype} values.

\begin{table}[htp]
    \centering
    \caption{Call/return \textbf{itype} values and corresponding instructions}
    \label{tab:itype}
    \begin{tabulary}{\textwidth}{|l|l|p{80mm}|}
        \hline
        \textbf {Type} & \textbf {Value} & \textbf {Instructions} \\
        \hline
        Unpredictable call & 8 & \textit{\textbf {jalr}} x1/x5, rs unless preceded by \textit{\textbf {auipc}} rs, \textit{\textbf {lui}} rs or \textit{\textbf {c.lui}} rs\newline
\textit{\textbf {c.jalr}} rs1 unless preceded by \textit{\textbf {auipc}} rs1, \textit{\textbf {lui}} rs1 or \textit{\textbf {c.lui}} rs1\newline
\textit{\textbf {jalr}} x0, rs where rs !=x1/5, unless preceded by \textit{\textbf {auipc}} rs, \textit{\textbf {lui}} rs or \textit{\textbf {c.lui}} rs (tail call)\newline
\textit{\textbf {c.jr}} rs1 where rs1 !=x1/x5, unless preceded by \textit{\textbf {auipc}} rs1, \textit{\textbf {lui}} rs1 or \textit{\textbf {c.lui}} rs1 (tail call) \\
        \hline
        Predictable call & 9 & \textit{\textbf {jal}} x1/x5\newline
\textit{\textbf {c.jal}}\newline
\textit{\textbf {jalr}} x1/x5, rs preceded by \textit{\textbf {auipc}} rs, \textit{\textbf {lui}} rs or \textit{\textbf {c.lui}} rs\newline
\textit{\textbf {c.jalr}} rs1 preceded by \textit{\textbf {auipc}} rs1, \textit{\textbf {lui}} rs1 or \textit{\textbf {c.lui}} rs1\newline
\textit{\textbf {jalr}} x0, rs where rs !=x1/5 preceded by \textit{\textbf {auipc}} rs, \textit{\textbf {lui}} rs or \textit{\textbf {c.lui}} rs (tail call)\newline
\textit{\textbf {c.jr}} rs1 where rs1 !=x1/x5 preceded by \textit{\textbf {auipc}} rs1, \textit{\textbf {lui}} rs1 or \textit{\textbf {c.lui}} rs1 (tail call)\\
        \hline
        Unpredictable return & 10 & \textit{\textbf {jalr}} x0, x1/x5 unless preceded by \textit{\textbf {auipc}} x1/x5, \textit{\textbf {lui}} x1/x5 or \textit{\textbf {c.lui}} x1/x5\newline

\textit{\textbf {c.jr}} x1/x5 unless preceded by \textit{\textbf {auipc}} x1/x5, \textit{\textbf {lui}} x1/x5 or \textit{\textbf {c.lui}} x1/x5\\
        \hline
        Predictable return & 11 & \textit{\textbf {jalr}} x0, x1/x5 preceded by \textit{\textbf {auipc}} x1/x5, \textit{\textbf {lui}} x1/x5 or \textit{\textbf {c.lui}} x1/x5\newline
\textit{\textbf {c.jr}} x1/x5 preceded by \textit{\textbf {auipc}} x1/x5, \textit{\textbf {lui}} x1/x5 or \textit{\textbf {c.lui}} x1/x5\\
        \hline
        Unpredictable non call/return jump & 12 & \textit{\textbf {jalr}}, \textit{\textbf {c.jalr}}, \textit{\textbf {c.jr}} not matching any of the above\\
        \hline
        Predictable non call/return jump & 13 & \textit{\textbf {jal}} not matching any of the above\\
        \hline
    \end{tabulary}
\end{table}

Table~\ref{tab:context-type} specifies the actions for the various \textbf{context\_type} values.

\begin{table}[htp]
    \centering
    \caption{Call/return \textbf{context\_type} values and corresponding actions}
    \label{tab:context-type}
    \begin{tabulary}{\textwidth}{|l|l|p{80mm}|}
        \hline
        \textbf {Type} & \textbf {Value} & \textbf {Actions} \\
        \hline
        Context change with discontinuity & 0 &  An example would be a change of HART.\newline
        Need to report the last instruction executed on the previous context, as well as the 1st on the new context.\newline
        This is basically treated the same as an exception\\
        \hline
        Precise context change & 1 & Need to output the address of the 1st instruction, and the new context.\newline
        If there were unreported branches beforehand, these need to output those first.\newline
        Treated the same as a privilege change\\
        \hline
        Imprecise context change & 2 & An example would be a SW thread change.\newline
        Report the new context value at the earliest convenient opportunity.\newline
        It is reported without any address information, and the assumption is that the precise point of context change can be deduced from the source code (e.g. a CSR write). \\
        \hline
        Notification & 3 & An example would be a watchpoint.\newline
        Need to output the address of the watchpoint instruction.\newline
        Similar to an updiscon.\newline
        The context itself is not output\\
        \hline
    \end{tabulary}
\end{table}

\subsection {Example Signal Blocks}

\begin{table}[htp]
    \centering
    \caption{Example 1 : 9 Instructions retired in one cycle, 3 branches} 
    \label{tab:signal-block-9-instructions-3-branches}
    \begin{tabulary}{\textwidth}{|l|l|}
        \hline
        \textbf {Retired} & \textbf {Instruction Trace Bundle} \\
        \hline
        1000: \textbf {\textit{divuw}} &  \textbf {iretire}=7, \textbf {iaddr}=0x1000, \textbf {ntkn}=0, \textbf {itype}=4\\
        1004: \textbf {\textit{add}} &  \\
        1008: \textbf {\textit{or}} &  \\
        100C: \textbf {\textit{c.jalr}} &  \\
        \hline
        0940: \textbf {\textit{addi}} &  \textbf {iretire}=4, \textbf {iaddr}=0x0940, \textbf {ntkn}=1, \textbf {itype}=6\\
        1004: \textbf {\textit{c.beq}} &  \\
        1008: \textbf {\textit{c.bnez}} &  \\
        \hline
        0988: \textbf {\textit{lbu}} &  \textbf {iretire}=4, \textbf {iaddr}=0x0988, \textbf {ntkn}=0, \textbf {itype}=0\\
        098C: \textbf {\textit{csrrw}} &  \\
        \hline
    \end{tabulary}
\end{table}


\subsection {Side band signals}

In some circumstances there will be some side band signals which may
affect the encoder's behaviour, for example to start and/or stop
encoding.
There will sometimes be cases where the encoder may be
required to affect the behaviour of the core, for example stalling.

Note, any user defined information that needs to be output by the encoder
will need to be applied to the \textbf{context} value.

\begin{table}[htp]
    \centering
    \caption{User Sideband Encoder Ingress signals}
    \label{tab:ingress-side-band}
    \begin{tabulary}{\textwidth}{|l|l|}
        \hline
        \textbf {Signal} & \textbf {Function} \\
       \hline
        \textbf {user} [\textit{user\_width\_p}-1:0] &  Sideband signals \\
        \hline
        \textbf {halted}& Core is stalled or halted \\
        \hline
        \textbf {reset}& Core in reset \\
        \hline
    \end{tabulary}
\end{table}

\begin{table}[htp]
    \centering
    \caption{User Sideband Encoder Egress signals}
    \label{tab:engress-side-band}
    \begin{tabulary}{\textwidth}{|l|l|}
        \hline
        \textbf {Signal} & \textbf {Function} \\
        \hline
        \textbf {stall}& Stall request to core \\
        \hline
    \end{tabulary}
\end{table}


\subsection {Parameters}

The encoder will have some configurable or variable parameters. Some
of these are related to port widths whilst others may indicate the
presence or otherwise of various feature, e.g. filter or comparators.
Table~\ref{tab:parameters} outlines the list of parameters.

How the parameters are input to the encoder is implementation specific.  
\FloatBarrier
\begin{table}[h]
    \centering
    \caption{Parameters to the encoder}
    \label{tab:parameters}
    \begin{tabulary}{\textwidth}{|l|p{25mm}|p{80mm}|}
        \hline
        \textbf {Parameter name} & \textbf {Range} & \textbf {Description} \\
        \hline
        \textit {context\_width\_p} & 1-32 & Width of context bus \\
        \hline
        \textit {ecause\_width\_p} & 1-16 & Width of exception cause bus \\
        \hline
        \textit {iaddress\_lsb\_p} & 0-3 & LSB of instruction address bus \\
        \hline
        \textit {iaddress\_width\_p} & 2-128 & Width of instruction address bus. This is the same as \textit {XLEN}\\
        \hline
        \textit {nocontext\_p} & 0 or 1 & Ignore context if 1 \\
        \hline
        \textit {notval\_p} & 0 or 1 & Ignore trap value if 1 \\
        \hline
        \textit {privilege\_width\_p} & 1-4 & Width of privilege bus \\
        \hline
        \textit {ecause\_choice\_p} & 0-6 & Number of bits of exception cause to match using multiple choice \\
        \hline
        \textit {filter\_context\_p} & 0,1 & Filtering on context supported when 1 \\
        \hline
        \textit {filter\_ecause\_p} & 0-15 & Filtering on exception cause supported when non\_zero.  Number of nested exceptions supported is 2\textsuperscript{filter-ecause-p} \\
        \hline
        \textit {filter\_interrupt\_p} & 0,1 & Filtering on interrupt supported when 1 \\
        \hline
        \textit {filter\_privilege\_p} & 0,1 & Filtering on privilege supported when 1 \\
        \hline
        \textit {filter\_tval\_p} & 0,1 & Filtering on trap value supported when 1 \\
        \hline
        \textit {user\_width\_p} & 0-256 & Width of user-defined filter qualifier input bus \\
        \hline
        \textit {taken\_branches\_p} & 1-8 & Number of times \textbf{iretire}, \textbf{itype}, \textbf{ntkn} is replicated\\
        \hline
        \textit {itype\_width\_p} & 3-4 & Width of the \textbf{itype} bus\\
        \hline
        \textit {iretire\_width\_p} & 2-8 & Width of the \textbf{iretire} bus\\
        \hline
        \textit {ntkn\_width\_p} & 1-5 & Width of the \textbf{ntkn} bus\\
        \hline
        \textit {context\_type\_width\_p} & 2 & Width of the \textbf{context\_type} bus\\
        \hline
    \end{tabulary}
\end{table}
\FloatBarrier

\subsection {Discovery of parameter values}

The parameters used by the encoder must be discoverable at
runtime. Some external entity, for example a debugger or a supervisory
hart would issue a discovery command to the encoder. The encoder will
provide the discovery information as encapsulated in the following
parameters in one or more different formats.  The preferred format
would be in a packet which is sent over the trace infrastructure.

Another format would may be allowing the external enity to read the
values from some register or memory mapped space maintained by the
encoder.

\begin{itemize}
    \item \textit {minor\_revision}. Identifies the minor revision.
    \item \textit {version}. Identifies the module version.
    \item \textit {comparators}. The number of comparators is comparators+1.
    \item \textit {filters}. Number of filters is filters+1.
    \item \textit {context\_width}. Width of context input bus is \textit {context\_width}+1.
    \item \textit {ecause\_choice}. Number of LSBs of the ecause input bus that can be filtered using multiple choice.
    \item \textit {ecause\_width}. Width of the ecause input bus is \textit {ecause\_width}+1.
    \item \textit {filter\_context}. Filtering on the \textit {context} input bus supported when 1.
    \item \textit {filter\_ecause}. Filtering on the ecause input bus supported when non-zero.  Number of nested exceptions supported is $2^\frac{\textit {filter\_ecause}}{}$.
    \item \textit {filter\_interrupt}. Filtering on the interrupt input signal supported when 1.
    \item \textit {filter\_privilege}. Filtering on the privilege input bus supported when 1.
    \item \textit {filter\_tval}. Filtering on the tval input bus supported when 1.
    \item \textit {iaddress\_lsb}. LSB of iaddress output in trace encoder data messages.
    \item \textit {iaddress\_width}. Width of the iaddress input bus is \textit {iaddress\_width} + 1.
    \item \textit {nocontext}. Context ignored when 1.
    \item \textit {notval}. Trap value ignored when 1.
    \item \textit {privilege\_width}. Width of the privilege input bus is \textit {privilege\_width} + 1.
    \item \textit {rv32}. ISA is RV32 when 1.
    \item \textit {itype\_width}. Width of the \textbf{itype} bus.
    \item \textit {iretire\_width}. Width of the \textbf{iretire} bus.
    \item \textit {ntkn\_width}. Width of the \textbf{ntkn} bus.
    \item \textit {ilastsize\_width}. Width of the \textbf{ilastsize} bus.
    \item \textit {taken\_branches}. Number of times \textbf{iretire}, \textbf{itype}, \textbf{ntkn} is replicated
    \item \textit {context\_type\_width}. Width of the \textbf{context\_type} bus
\end{itemize}
