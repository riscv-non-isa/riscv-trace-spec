\chapter{Filtering} \label{ch:filtering}


The instruction trace encoder must be able to filter on the following inputs to the encoder:

\begin{itemize}
  \item The instruction address 
  \item The context 
  \item The exception cause 
  \item Whether the exception is an interrupt or not 
  \item The privilege level
  \item Tval 
  \item User specific signals
\end{itemize}

Internal to the encoder will be several comparators and filters. The
actual number of these will vary for different classes of devices. The
filters and comparators must be configured to provide the trace and
filtering required. There will be three command types needed to set up
the filtering operation.

\begin{enumerate}
    \item Set up comparator 
    \begin{itemize}
      \item Which input bus to compare
        \begin {enumerate}
          \item address
          \item context
          \item tval
        \end{enumerate}
      \item Which comparator(s) to use which filtering operation to enable
        \begin {enumerate}
          \item \textit {eq}
          \item \textit {neq}
          \item \textit {lt}
          \item \textit {lte}
          \item \textit {gt}
          \item \textit {gte}
          \item \textit {always}
          \end{enumerate}
    \end{itemize}

  \item Value e.g. start address
    \item Set up filter 
    \item Set match 
    \begin{itemize}
      \item Configure matching behaviour for exception, privilege and user sideband
    \end{itemize}
\end{enumerate}

The user may wish to:

\begin{enumerate}
  \item Trace instructions between a range of addresses
  \item Trace instruction from one address to another
  \item Trace interrupt service routine
  \item Start/stop trace when in a particular privilege 
  \item Start/stop trace when context changes or is a particular value
  \begin {itemize}
    \item This can be HARTs and/or software contexts. If the latter this would be 
    \item Start/stop trace when specific instruction
    \item Start/stop based on \textbf{user} sideband signals 
    \item This could be the specific CSR value being presented to the Encoder
  \end{itemize}
\end{enumerate}

\section {Using trigger outputs from Debug Module}

The debug module of the RISC-V core may have a trigger unit. This exposes a 4-bit field as shown 
in table \ref{tab:debugModuleTriggerSupport}.

\begin{table}[!h]
    \centering
    \caption{Debug module trigger support (mcontrol)}
    \label{tab:debugModuleTriggerSupport}
    \begin{tabulary}{\textwidth}{|l|p{80mm}|}
        \hline
        \textbf {Value} & \textbf {Description} \\
        \hline
        2 & Trace on\\
        \hline
        3 & Trace off \\
        \hline
        4 & Trace single. The 'single' action for an instruction
        trigger could cause just that instruction to be traced if connected to a \textbf{user} input;
        Alternatively it could be used to assert the 'Notification' \textbf{context\_type} to generate
        a watchpoint trace. \\
        \hline
    \end{tabulary}
\end{table}
